\chapter*{CHAPTER 3: INTERNSHIP ACTIVITIES}
\addcontentsline{toc}{chapter}{CHAPTER 3: INTERNSHIP ACTIVITIES}

\setcounter{chapter}{3}
\setcounter{section}{0} 
This chapter details the tasks, responsibilities and technical activities performed during the internship. It includes an overview of assigned duties, a weekly log of activities, and description of projects worked on.
\section{Roles and Responsibilities}
\subsection{Roles}
\begin{itemize}
    \item Frontend Intern
\end{itemize}
\subsection{Responsibilities}
During the internship, the following responsibilities were undertaken:

\begin{itemize}
    \item Designing and developing responsive user interfaces using modern frontend technologies such as React.js and Next.js.
    \item Converting UI/UX designs into functional web pages while ensuring visual accuracy and cross-browser compatibility.
    \item Implementing reusable components, managing application state, and organizing frontend code for scalability and maintainability.
    \item Integrating frontend applications with backend APIs to enable dynamic data handling and real-time functionality.
    \item Testing, debugging, and optimizing frontend features to improve performance, responsiveness, and overall user experience.
\end{itemize}


\section{Weekly Log}
\begin{longtable}{|p{2.5cm}|p{11cm}|}
\caption{Weekly Tasks Performed During Internship} \\
\hline
\textbf{Week} & \textbf{Description of Tasks Performed} \\ \hline
\endfirsthead
\hline
\textbf{Week} & \textbf{Description of Tasks Performed} \\ \hline
\endhead
\endfoot
\hline
\endlastfoot

Week 1 (Sep 22-27) & \tabitems{
    \item Learned React Basic Fundamentals
    \item Learned best practices in JS programming
    \item Learned tailwind CSS
} \\ \hline

Week 2 (Oct 6-10) & \tabitems{
    \item Learned React Hooks: useState and useEffect
    \item Learned Custom Hooks in React and its useCases
    \item Implemented useState and useEffect for state management
    \item Created Custom Hook counter to keep track of count of button pressed
} \\ \hline

Week 3 (Oct 13-17) & \tabitems{
    \item Completed frontend training exercises in React.js, JavaScript, and modern tools
    \item Developed basic components: loading bar, toast notifications, loading skeletons
    \item Learned Git and GitHub for version control
    \item Built a weather application using public API with TanStack Query
} \\ \hline

Week 4 (Oct 27-31) & \tabitems{
    \item Implemented Hoslog Affiliater Landing Page from Figma design
    \item Developed page using React.js and vanilla CSS
    \item Ensured mobile responsiveness and cross-device compatibility using CSS media queries
} \\ \hline

Week 5 (Nov 3-7) & \tabitems{
    \item Identified and resolved UI/UX issues reported by the design team
    \item Fixed layout, responsiveness, and interaction-related bugs
    \item Re-tested website across different devices to ensure stability
} \\ \hline

Week 6 (Nov 10-14) & \tabitems{
    \item Learned React Hook Form and Zod
    \item Implemented form validation using React Hook Form + Zod
    \item Designed and developed a clone of a Blog site
    \item Collaborated with fellow interns using Git for team workflows
    \item Implemented UI features: pagination, scroll to top, image animations
} \\ \hline

Week 7 (Nov 17-21) & \tabitems{
    \item Created a landing page for an online Exam site
    \item Create exam page like Multiple choice question and essay section
    \item Implemented pagination and routing for the webpage
} \\ \hline

Week 8 (Nov 24-28) & \tabitems{
    \item Developed Square Japan Landing Page from Figma design
    \item Created Recruitment section
    \item Implemented inquiry forms using React Hook Form + Zod
    \item Integrated Shadcn UI components for cards
} \\ \hline

Week 9 (Dec 1-5) & \tabitems{
    \item Developed Tapaiko Bazar e-commerce website
    \item Fix issue in the Square Japan Landing Page 
    \item Built admin portal using React.js + Shadcn UI; client portal using Next.js + Shadcn UI
    \item Created reusable and scalable components
    \item Designed admin vendor table, profile page, single product page, and login page using dummy data
} \\ \hline

Week 10 (Dec 8-12) & \tabitems{
    \item Consumed and integrated backend APIs for dynamic functionality
    \item Implemented admin-side features: login, add-product, profile, vendors
    \item Implemented admin-side APIs for profile, product, and vendor management
} \\ \hline

\end{longtable}

\section{Tasks/Activities Performed}

\subsection*{1. Hoslog Affiliate Landing Page}

The Hoslog Affiliate Landing Page was developed using React.js in combination with vanilla CSS. The primary objective of the project was to translate a predefined Figma design into a fully functional and responsive web application. Particular emphasis was placed on maintaining visual fidelity to the original design, ensuring cross-device compatibility through the use of CSS media queries, and enhancing overall user experience. The development process incorporated responsive layout techniques and interactive UI components to improve usability. Additionally, the landing page was tested across multiple web browsers to ensure consistent performance and functionality in different browsing environments.    

\begin{figure}[p]
    \centering
    \includegraphics[width=\linewidth,height=0.9\textheight,keepaspectratio]{figure/Affiliate/code.png}
    \caption{Code Snippet of Hoslog Affiliate Landing Page}
    \label{fig:hoslog-landing-page-full}
\end{figure} 

\begin{figure}[htbp]
    \centering
     \includegraphics[width=1\linewidth]{\detokenize{figure/Affiliate/Screenshot 2026-01-09 133757.png}}
    \caption{Hoslog Affiliate Landing Page Hero Section}
    \label{fig:hoslog-landing-page}
 \end{figure}

 \begin{figure}[htbp]
    \centering
     \includegraphics[width=1\linewidth]{\detokenize{figure/Affiliate/Screenshot 2026-01-09 133806.png}}
    \caption{Hoslog Affiliate Landing Page Features Section}
    \label{fig:hoslog-landing-page-features}
 \end{figure}

\begin{figure}[htbp]
    \centering
    \includegraphics[width=1\linewidth]{\detokenize{figure/Affiliate/FAQ.png}}        
    \caption{Hoslog Affiliate Landing Page FAQ Section}
    \label{fig:hoslog-landing-page-faq}
\end{figure}

 \begin{figure}[htbp]
    \centering
     \includegraphics[width=1\linewidth]{\detokenize{figure/Affiliate/Screenshot 2026-01-09 133939.png}}
    \caption{Hoslog Affiliate Landing Page Footer Section}
    \label{fig:hoslog-landing-page-footer}
 \end{figure}

 \clearpage
\subsection*{2. Blog Site Clone}

The Blog Site Landing Page comprises three primary features developed using React with TypeScript. The Homepage implements pagination functionality with six articles per page, utilizing React Router's searchParams API to maintain state and Tailwind CSS media queries for responsive design across devices. The ScrollToTop utility component monitors route changes via the useLocation hook and automatically scrolls the viewport to the page top with smooth behavior, enhancing navigation experience in the single-page application. The Signup page integrates schema-based form validation using Zod and React Hook Form, capturing user credentials with real-time error feedback and asynchronous submission handling, providing a seamless user registration experience with confirmation messaging and navigation capabilities.
    \begin{figure}[p]
                \centering
                \includegraphics[width=\linewidth,height=0.9\textheight,keepaspectratio]{\detokenize{figure/blog/customcomponent.png}}
                \caption{Code Snippet of Custom Hook for Scroll to Top}
                \label{fig:blog-site-custom-hook}
        \end{figure}
\begin{figure}[p]
    \centering
     \includegraphics[width=\linewidth,height=0.9\textheight,keepaspectratio]{\detokenize{figure/blog/blogcode.png}}
    \caption{Code Snippet of Blog Site Clone}
    \label{fig:blog-site-code}
 \end{figure}

  
\begin{figure}[htbp]
    \centering
    \includegraphics[width=0.9\linewidth]{\detokenize{figure/blog/bloghome.png}}
    \caption{Blog Site Homepage}
    \label{fig:blog-site-homepage}
\end{figure}

\begin{figure}[htbp]
    \centering
    \includegraphics[width=0.9\linewidth]{\detokenize{figure/blog/signup.png}}
    \caption{Blog Site Signup Page}
    \label{fig:blog-site-signup}
\end{figure}

    \clearpage
\subsection*{3. Online Exam Site}
The Online Exam Application utilizes a structured navigation system built with React Router v6 and TypeScript to ensure a seamless and predictable user journey. At the core of the architecture is a centralized routing component that wraps the entire application in a dedicated exam provider. This setup uses the React Context API to maintain a consistent global state, allowing critical data—such as user responses, remaining time, and overall progress—to persist even as the student moves between different sections of the test.

The navigation logic is designed to guide users through a linear examination flow, starting with an automatic redirect from the root path to the first assessment module. By using programmatic navigation with history management, the system prevents users from accidentally navigating back into an empty state, ensuring the integrity of the testing session. The application is divided into four primary segments—Listening, Reading, Writing, and a final Summary—each mapped to a specific URL path to maintain a clear and organized single-page structure.

By leveraging a component-based routing approach paired with TypeScript, the system achieves high type safety and prevents common navigation errors. This ensures that as a student progresses from listening comprehension through to the final writing assessment, the transition is fluid and the underlying data remains synchronized. The final summary interface serves as a comprehensive conclusion to the workflow, pulling all stored state together to provide a clear overview of the completed examination.
\begin{figure}[p]
    \centering
     \includegraphics[width=\linewidth,height=0.9\textheight,keepaspectratio]{\detokenize{figure/OnlineExam/listeningcode.png}}
    \caption{Code Snippet of Online Exam Site Routing}
    \label{fig:online-exam-site-code}
\end{figure}
\begin{figure}[htbp]
    \centering
     \includegraphics[width=1\linewidth]{\detokenize{figure/OnlineExam/examlistening.png}}
    \caption{Online Exam Site Listening Section}
    \label{fig:online-exam-site-listening}
\end{figure}
\begin{figure}[p]
    \centering
     \includegraphics[width=\linewidth,height=0.9\textheight,keepaspectratio]{\detokenize{figure/OnlineExam/writingcode.png}}
    \caption{Online Exam Site Speaking Section}
    \label{fig:online-exam-site-speaking}
\end{figure}
\begin{figure}[htbp]
    \centering
     \includegraphics[width=1\linewidth]{\detokenize{figure/OnlineExam/examwriting.png}}
    \caption{Online Exam Site Writing Section}
    \label{fig:online-exam-site-writing}
\end{figure}
\begin{figure}[p]
    \centering
     \includegraphics[width=\linewidth,height=0.9\textheight,keepaspectratio]{\detokenize{figure/OnlineExam/readingcode.png}}
    \caption{Code Snippet of Online Exam Site Summary Section}
    \label{fig:online-exam-site-summary-code}
\end{figure}

\begin{figure}[htbp]
    \centering
     \includegraphics[width=1\linewidth]{\detokenize{figure/OnlineExam/examreading.png}}
    \caption{Online Exam Site Reading Section}
    \label{fig:online-exam-site-reading}
\end{figure}
\clearpage
\subsection*{4. Square Japan Landing Page}
The Square Japan landing page was developed using Next.js, meticulously translating a Figma design into a high-performance web experience. To increase user engagement, the header section incorporates smooth, timed animations that activate upon page load or as the user scrolls, creating a polished and interactive first impression. These transitions provide a modern feel while guiding the user’s attention naturally toward the primary messaging of the site.

Information architecture was a key focus, particularly within the FAQ section. Utilizing Shadcn UI components, this area features an accordion-based layout to keep the interface clean while providing immediate answers to common questions. To further improve usability, a tabbed navigation system was integrated, allowing users to switch between different FAQ categories effortlessly. This ensures that even large volumes of information remain organized and accessible without cluttering the landing page.

The communication and recruitment workflows were designed for maximum reliability and brand consistency. A dedicated modal dialog was implemented for the “Join Our Team” feature, offering a streamlined path for applicants to submit their details. Behind the scenes, custom email templates were developed for both general inquiries and recruitment forms to ensure all outreach reflects the professional branding of the project. These workflows are powered by the Resend email service, which automates confirmation messages to provide users with immediate and reliable feedback upon every submission.\begin{figure}[htbp]
    \centering
     \includegraphics[width=1\linewidth]{\detokenize{figure/Report/inquirycode.png}}
    \caption{Code Snippet of Square Japan inquiry Page}
    \label{fig:square-japan-inquiry-code}
\end{figure}
\begin{figure}[p]
    \centering
     \includegraphics[width=\linewidth,height=0.9\textheight,keepaspectratio]{\detokenize{figure/Report/inquirycode.png}}
    \caption{Code snippet of the Square Japan inquiry page (full view)}
    \label{fig:square-japan-inquiry-code-full}
\end{figure}
\begin{figure}[p]
    \centering
     \includegraphics[width=\linewidth,height=0.9\textheight,keepaspectratio]{\detokenize{figure/Report/inquiry.png}}
    \caption{Code Snippet of Square Japan Landing Page}
    \label{fig:square-japan-landing-page-code}
\end{figure}
\clearpage
\subsection*{4. Tapaiko Bazar E-commerce Website}
Tapaiko Bazar project involved building a comprehensive, web-based dashboard system from the ground up. Developed through a close partnership between frontend and backend teams, the platform features distinct interfaces for administrators and clients. Our goal was to create a secure, scalable, and user-friendly ecosystem that simplifies vendor and product management while ensuring a smooth experience for everyday users.

The administrative side features a structured authentication system designed for security and ease of use. A clean, card-based login interface utilizes Zod for schema-based validation, ensuring that email and password inputs are accurate before submission.

To manage credential recovery, a modular "Forgot Password" workflow was implemented. This process follows a three-step sequence—email submission, one-time password (OTP) verification, and a secure password reset. Each step is handled through dedicated API endpoints, maintaining a clear separation between frontend form handling and backend security logic.
Efficient vendor and product management are central to the administrative workflow. The vendor management module utilizes a responsive data table and Tailwind CSS to display records clearly across all devices. Similarly, the product management interface employs a two-column layout that separates product metadata from media management. By using the FormData API for image uploads and implementing automatic input resets upon successful submission, the system maintains a clean and high-performance environment for handling complex product data and multi-attribute variations.

On the client-facing side, the platform emphasizes intuitive navigation and discovery. A dynamic filtering system allows for product searches based on tags and price ranges, with states synchronized to URL parameters to ensure persistence during navigation. To further enhance the shopping experience, an intelligent variant selection algorithm evaluates product combinations in real-time, preventing users from selecting unavailable options. This is complemented by a custom carousel wrapper built for smooth image galleries and a dedicated website review module that displays overall platform feedback through dynamic star ratings and optimized imagery.

The platform maintains high standards of consistency by mirroring security workflows across all user roles. The password recovery process for clients utilizes the same multi-step modal design as the administrative side, providing a familiar and secure method for regaining account access. By integrating reusable UI components and synchronized backend services, the system ensures real-time data accuracy and a cohesive user experience throughout the entire application.

\begin{figure}[p]
    \centering
     \includegraphics[width=\linewidth,height=0.9\textheight,keepaspectratio]{\detokenize{figure/ecommerce/loginform.png}}
    \caption{Tapaiko Bazar Admin Login Interface}
    \label{fig:tapaiko-login}
\end{figure}

\begin{figure}[p]
    \centering
     \includegraphics[width=\linewidth,height=0.9\textheight,keepaspectratio]{\detokenize{figure/ecommerce/vendorform.png}}
    \caption{Tapaiko Bazar Vendor Form Module}
    \label{fig:tapaiko-vendor-form}
\end{figure}

\begin{figure}[p]
    \centering
     \includegraphics[width=\linewidth,height=0.9\textheight,keepaspectratio]{\detokenize{figure/Report/vendortable.png}}
    \caption{Vendor Table Overview}
    \label{fig:tapaiko-vendor-table}
\end{figure}

\begin{figure}[p]
    \centering
     \includegraphics[width=\linewidth,height=0.9\textheight,keepaspectratio]{\detokenize{figure/Report/productdetails.png}}
    \caption{Product Details and Management View}
    \label{fig:tapaiko-product-details}
\end{figure}

\begin{figure}[p]
    \centering
     \includegraphics[width=\linewidth,height=0.9\textheight,keepaspectratio]{\detokenize{figure/Report/productmanagement.png}}
    \caption{Product Management Dashboard}
    \label{fig:tapaiko-product-management}
\end{figure}

\begin{figure}[p]
    \centering
     \includegraphics[width=\linewidth,height=0.9\textheight,keepaspectratio]{\detokenize{figure/Report/filtereddata.png}}
    \caption{Filtered Product Data View}
    \label{fig:tapaiko-filtered-data}
\end{figure}

\begin{figure}[p]
    \centering
     \includegraphics[width=\linewidth,height=0.9\textheight,keepaspectratio]{\detokenize{figure/Report/productcart.png}}
    \caption{Client Portal Product Cart Interface}
    \label{fig:tapaiko-product-cart}
\end{figure}

\begin{figure}[p]
    \centering
     \includegraphics[width=\linewidth,height=0.9\textheight,keepaspectratio]{\detokenize{figure/Report/websitereview.png}}
    \caption{Website Review and Rating Component}
    \label{fig:tapaiko-website-review}
\end{figure}

\begin{figure}[p]
    \centering
     \includegraphics[width=\linewidth,height=0.9\textheight,keepaspectratio]{\detokenize{figure/ecommerce/productmanagement.png}}
    \caption{Tapaiko Bazar Product Management UI}
    \label{fig:tapaiko-productmanagement-ui}
\end{figure}

\begin{figure}[p]
    \centering
     \includegraphics[width=\linewidth,height=0.9\textheight,keepaspectratio]{\detokenize{figure/ecommerce/productslider.png}}
    \caption{Product Slider with Media Gallery}
    \label{fig:tapaiko-product-slider}
\end{figure}

\begin{figure}[p]
    \centering
     \includegraphics[width=\linewidth,height=0.9\textheight,keepaspectratio]{\detokenize{figure/ecommerce/varianthandleinproduct.png}}
    \caption{Variant Handling in Product Creation}
    \label{fig:tapaiko-variant-handling}
\end{figure}



