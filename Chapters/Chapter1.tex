\chapter*{CHAPTER 1: INTRODUCTION}
\addcontentsline{toc}{chapter}{CHAPTER 1: INTRODUCTION}
\setcounter{chapter}{1}
\section{Introduction}
In the beginning, one of the projects assigned to interns at Global Square IT Company was developing the landing page for Affilater Hoslog. The main objective was to convert a predefined Figma design into a fully functional and visually accurate website. The landing page was developed using React.js and vanilla CSS, with special emphasis on responsive layouts and mobile compatibility through CSS media queries. This project provided interns with practical exposure to translating design specifications into a dynamic and interactive user interface.

Another important project was the creation of a landing page for Square Japan. This project included interactive sections such as a recruitment page and an inquiry form. Next.js was used to build the landing page, and Tailwind CSS was applied to ensure consistent, maintainable, and responsive styling.

The final project assigned to the interns was a full-featured collaborative e-commerce web application. This involved designing and implementing the front-end UI for both customer and admin portals. On the admin side, the dashboard was developed with a sidebar and data visualization charts using Recharts to provide actionable insights. On the client side, key pages such as the login page and product listing pages were developed to ensure a smooth user experience. The e-commerce application was built using React.js and Next.js for the frontend, with CSS and Tailwind CSS for styling. The project emphasized responsive design, user experience, and integration with backend APIs to create a functional and engaging online shopping platform.
\section{Problem Statement}
With the increasing demand for seamless online experiences, organizations need web applications that are both visually attractive and functionally robust. Frontend development involves several difficulties, such as converting design mockups into functional interfaces, managing dynamic data, ensuring cross-device compatibility, and maintaining code quality for long-term scalability. The internship mainly focused on addressing these challenges by working on real-world projects, implementing features like authentication, dynamic product displays, responsive layouts, and interactive components, thereby bridging the gap between academic learning and practical frontend development experience.

\section{Objective}
The goals of the internship  were as follows:
\begin{itemize}
    \item To follow the industry's best practices in front-end development by writing clean, modular, and maintainable code, performing UI testing, and using version control tools to support collaboration
    \item To integrate frontend components with backend services, implement dynamic features, and enhance application performance, usability, and maintainability through real-world project development.
\end{itemize}

\section{Scope and Limitations}
\subsection{Scope}
The scope of the E-commerce web application developed during the internship includes the following:


\begin{itemize}
\item The internship involved creating front-end interfaces for various web applications, including landing pages and e-commerce platforms, utilizing modern technologies such as React.js, Next.js, CSS, and Tailwind CSS.

\item The work emphasized responsive design, user experience, integration with backend APIs, and development of dynamic features to improve functionality and interactivity.

\end{itemize}
\subsection{Limitations}
Despite its functionality and practical implementation, the E-commerce web application developed during the internship faces the following technical and operational limitations:

\begin{itemize}
    
\item The internship was mainly centered on frontend development, providing limited experience with advanced backend processes, database management, and server-side optimization.

\item Certain projects were restricted by predefined designs, tight deadlines, and dependency on backend-developed APIs, which constrained full end-to-end control over the system.

\end{itemize}

\section{Report Organization}
The report is organized in the systematic pattern where each chapter includes the subtopics to support the main title of the report. The report is organized as follows: 

\begin{itemize}
    \item \textbf{Chapter 1: Introduction} \\
    The chapter initiates with the introduction of the whole report, systematically partitioned into four distinct subchapters. It includes introduction of the internship project, problem statement, objectives, scope and limitation. 

    \item \textbf{Chapter 2: Organizational Review and Literature Review} \\
    This chapter offers a concise overview of organization involved in the project and includes a review of relevant literature related to the project's domain.
    \clearpage
    \item \textbf{Chapter 3: Internship Activities} \\
    This chapter includes the roles, responsibilities and technical activities undertaken during the internship in detail. It also covers descriptions of the projects involved and a comprehensive list of tasks and activities completed. 

    \item \textbf{Chapter 4: Conclusion and Learning Outcomes} \\
    This chapter summarizes the project's architectural design, details the data utilized, and provides insights into the overall execution process of the project.
\end{itemize}